\magnification=1050 \baselineskip=12pt plus 1pt minus 2pt
\nopagenumbers


\noindent {\bf Manuscript Code Number: DN12236}

\smallskip
\noindent {\bf Title: Scattering from compact objects: Regge poles and the Complex Angular Momentum method}

\smallskip
\noindent {\bf Authors: Mohamed Ould El Hadj, Tom Stratton and Sam R. Dolan}




\bigskip
\bigskip

We thank the referee for her/his careful reading of our manuscript and her/his useful remarks that have been taken into account and that have permitted us to improve our article:


\bigskip

\noindent $\underline{\hbox{Comments:}} $
\medskip


\bigskip
(i) In the case where $\alpha \rightarrow0$, the effective potential becomes continuous and the branch of Regge poles associated with the discontinuity, i.e. the broad resonances ``disappears''. However, this does not mean that the black hole does not have a pole. Because the effective potential of the black hole is not only continuous, it has a maximum and the poles of the $\cal{S}$--matrix (QNFs and/or RPs) are associated with this maximum (see for example Folacci, Decanin and Jensen: Complex angular momentum in black hole physics and quasinormal modes and reference therein. Ref[40] in our article). 

\bigskip 
For the issue of continuity it is worth distinguishing between two cases: \bigskip

\qquad  (a)  in our model, $\alpha \rightarrow 0$ iff $M/R \rightarrow 0$, e.g. a large dilute star. In this case the modes move close to the real axis.
In the case $\alpha = 0$, $M/R = 0$, there is no star at all and no modes at all. (cf. singular perturbation theory).

\bigskip

 \qquad (b) In different stellar models (such as a polytrope) in which the potential is $C^n$ at the surface ($n \geq 0$), the imaginary part of the broad resonance branch is proportional to $(n+1)$ and in the large-$n$ limit the associated modes have large imaginary parts and little physical consequence.




\bigskip
\bigskip
(ii) The referee is correct that the radius of the body does enter the approximation scheme in eqs 30 and 31. However, the Wronskian (eq. 29) is evalutated at  $r_{*} \approx 1/\omega$, and thus terms of  $O(M r_{*}^2 / R^3  )$ are of  $ O \left(  (M \omega)^{-2} (M/R)^3 \right) $ and we have $M/R< 1/2$. Hence, the  scale $M \omega$ is really the controlling factor in the expansion scheme. On the other hand, we must have $R/M>3$ for the WKB approximation to be valid, and we have made this clearer in the text. 

%Perhaps one can refer to the use of the approximation Eq.(30) and (31) of the function f(r). The WKB approximation  in this case is valid just for $R > 3M$ i.e. in case the potential has no well. 



\bigskip
\bigskip
(iii)  Usually we stop when the sum over the Regge poles converges. For the background integral, its evaluation allows us to see whether it is negligible or not. For example, in the case of the black hole and ultra-compact objects, in the high frequency regime, the background integral is negligible. This is surprising since this is not the case in other areas of physics (acoustics, optics, electromagnetics, quantum mechanics,...), where the CAM method finds its origins, and this independently of the frequency. To obtain a qualitative argument and deeper understanding of the Regge pole sum convergence, and significance of the background integral is desirable as the referee points out. Our article is focused on a proof of concept, and we have deferred this to further work.  


\bigskip\bigskip


\noindent $\underline{\hbox{Summary of changes made in response to referee's comments:}}  $
\medskip

\smallskip
(i) Added two paragraphs in section III D (after eq. 40) to discuss the limit $\alpha \rightarrow0$, different stellar models with continuous potentials at the surface, and clarifying that the analysis of this section is not applicable to black holes. 

\bigskip
(ii)  At the end of paragraph 2 in section III D we have clarified that the analysis is for models with $R/M>3$, and recalled this at the end of the section. 

\bigskip
(iii) Added a (penultimate) paragraph in section V expanding on comment (iii) above. 






\bigskip\bigskip


\noindent $\underline{\hbox{Other changes:}}  $
\medskip

\smallskip
-Added a reference: Sebastian H. V{\"o}lkel and Kostas D. Kokkotas, `On the Inverse Spectrum Problem of Neutron Stars'.
\bigskip
-Corrected a couple of minor typos. 



\vfill \eject \bye





