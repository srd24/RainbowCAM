\magnification=1050 \baselineskip=12pt plus 1pt minus 2pt
\nopagenumbers


\noindent {\bf Manuscript Code Number: DN12236}

\smallskip
\noindent {\bf Title: Scattering from compact objects: Regge poles and the Complex Angular Momentum method}

\smallskip
\noindent {\bf Authors: Mohamed Ould El Hadj, Tom Stratton and Sam R. Dolan}




\bigskip
\bigskip

We thank the referee for her/his careful reading of our manuscript and her/his useful remarks that have been taken into account and that have permitted us to improve our article:


\bigskip

\noindent $\underline{\hbox{Comments:}} $
\medskip


\bigskip
\qquad - In the case where $α \rightarrow0$, the effective potential becomes continuous and the branch of Regge poles associated with the discontinuity, i.e. the broad resonances ``disappears''. However, this does not mean that the black hole does not have a pole. Because the effective potential of the black hole is not only continuous, it has a maximum and the poles of the $\cal{S}$--matrix (QNFs and/or RPs) are associated with this maximum (see for example Folacci, Decanin and Jensen: Complex angular momentum in black hole physics and quasinormal modes and reference therein. Ref[40] in our article).


\bigskip
\qquad - Perhaps one can refer to the use of the approximation Eq.(30) and (31) of the function f(r). The WKB approximation  in this case is valid just for $R > 3M$ i.e. in case the potential has no well. 



\bigskip
\qquad -  Usually we stop when the sum over the Regge poles converges. For the background integral, its evaluation allows us to see whether it is negligible or not. For example, in the case of the black hole and ultra-compact objects, in the high frequency regime, the background integral is negligible. This is surprising since this is not the case in other areas of physics (acoustics, optics, electromagnetics, quantum mechanics,...), where the CAM method finds its origins, and this independently of the frequency.


\bigskip\bigskip


\noindent $\underline{\hbox{Other changes:}}  $
\medskip

\smallskip
\qquad - Do not forget to add the Ref[Sebastian H. Völkel and Kostas D. Kokkotas] :  On the Inverse Spectrum Problem of Neutron Stars.



%@article{Volkel:2019gpq,
%      author         = "Völkel, Sebastian H. and Kokkotas, Kostas D.",
%      title          = "{On the Inverse Spectrum Problem of Neutron Stars}",
%      journal        = "Class. Quant. Grav.",
%      volume         = "36",
%      year           = "2019",
%      number         = "11",
%      pages          = "115002",
%      doi            = "10.1088/1361-6382/ab186e",
%      eprint         = "1901.11262",
%      archivePrefix  = "arXiv",
%      primaryClass   = "gr-qc",
%      SLACcitation   = "%%CITATION = ARXIV:1901.11262;%%"
%}



\vfill \eject \bye





