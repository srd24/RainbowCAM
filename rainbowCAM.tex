\documentclass[aps,prd,longbibliography,reprint,twocolumn,amsmath,amssymb,amsfonts,showpacs,superscriptaddress]{revtex4-1}%reprint

\linespread{1.2}


\usepackage{graphicx, epsfig, amssymb}
\usepackage{amsmath, amsfonts}
\usepackage{bm}
\usepackage{enumitem}
\usepackage[usenames]{color}
\definecolor{navyblue}{rgb}{0.0, 0.0, 0.5}
\usepackage[linktocpage,colorlinks=true,allcolors=navyblue]{hyperref}
\usepackage[caption=false]{subfig}
\usepackage[utf8]{inputenc}
\usepackage{tensor}
\usepackage{soul}
%\usepackage{enumerate}
\usepackage{pict2e}
\usepackage{diagbox}
\usepackage{empheq}

%\usepackage[margin=1in]{geometry}
\usepackage{float}
%\usepackage{subfig}
\usepackage{comment}
\usepackage{array}
%\usepackage{varioref}
%\usepackage[hidelinks]{hyperref}
%\usepackage{cleveref}
%\usepackage[numbers]{natbib}
\usepackage{units}
\usepackage{wasysym}
%\usepackage{caption}
%\usepackage{subcaption}

\DeclareMathAlphabet{\mathpzc}{OT1}{pzc}{m}{it}
\newcommand{\ff}{\hat{f}}

\def\l{\left}
\def\r{\right}
\def\p{\partial}

\newcommand{\angarg}{(\theta,\phi)}
\newcommand{\dth}{\frac{d}{d \theta}}
\newcommand{\sh}{ {}_{-s} S_{\ell m}^{a \omega}}
\newcommand{\zh}{ {}_{-s} Z_{\ell m}^{a \omega}}
\newcommand{\syjm}{  {}_{-s}Y_{jm} }
\newcommand{\sylm}{  {}_{-s} Y_{\ell m} }
\newcommand{\sumjm}{\sum_{j=s}^{\infty} \sum_{m=-j}^{j}}
\newcommand{\sumjms}{\sum_{j,m}}
\newcommand{\cS}{\mathcal{S}}
\newcommand{\lmws}{{\ell m \omega s}}
\newcommand{\nn}{\nonumber}

\newcommand*\widefbox[1]{\fbox{\rule[-2cm]{0pt}{4cm}\hspace{2em}#1\hspace{2em}}}

\newcommand{\sam}[1]{\textcolor{red}{(Sam: #1)}}
\newcommand{\tom}[1]{\textcolor{blue}{(Tom: #1)}}
\newcommand{\mo}[1]{\textcolor{green}{(Mohamed: #1)}}

\renewcommand{\arraystretch}{1.3} % espace entre les lignes de la table
%\setlength{\tabcolsep}{.3cm} % espace entre les colonnes de la table

\usepackage{footnote}
\usepackage{tablefootnote}


\usepackage{threeparttable}


\begin{document}

\title{Scattering from compact objects, Regge poles \\ and the Complex Angular Momentum method}



\author{Mohamed \surname{Ould~El~Hadj}}
\email{m.ouldelhadj@sheffield.ac.uk}

\affiliation{Equipe Physique
Th\'eorique, SPE, UMR 6134 du CNRS
et de l'Universit\'e de Corse,\\
Universit\'e de Corse, Facult\'e des Sciences, BP 52, F-20250 Corte,
France}

\affiliation{Consortium for Fundamental Physics,  School of Mathematics and Statistics,
University of Sheffield, Hicks Building, Hounsfield Road, Sheffield S3 7RH, United Kingdom \looseness=-1}
%
\author{Tom Stratton}\email{tstratton1@sheffield.ac.uk}
\affiliation{Consortium for Fundamental Physics,  School of Mathematics and Statistics,
University of Sheffield, Hicks Building, Hounsfield Road, Sheffield S3 7RH, United Kingdom \looseness=-1}
%
\author{Sam R. Dolan}\email{s.dolan@sheffield.ac.uk}
\affiliation{Consortium for Fundamental Physics,  School of Mathematics and Statistics,
University of Sheffield, Hicks Building, Hounsfield Road, Sheffield S3 7RH, United Kingdom \looseness=-1}
%
\begin{abstract}
To be written
\end{abstract}

\date{\today}

\maketitle


\section{Introduction}
Blah. hello. some more text.

 The topic of time-independent scattering has been studied in detail since the 1960s \cite{Hildreth1964PhDT64, Matzner:1968, Vishveshwara:1970}, and there now exists a substantial literature \cite{Mashhoon:1973zz,Chrzanowski:1976jb,DeLogi:1977dp,Sanchez:1977vz,MatznerRyan1978,Handler:1980un,Matzner:1985rjn,Futterman:1988ni,Andersson:1995vi,Glampedakis:2001cx,Dolan:2006vj,Dolan:2007ut,Dolan:2008kf,Crispino:2009xt,Cotaescu:2014jca,Sorge:2015yoa,Gussmann:2016mkp,Leite:2017zyb,Nambu:2019sqn, Folacci:2019vtt,Leite:2019eis}.



\section{Differential scattering cross sections for scalar waves, their CAM representation and their Regge pole approximations}
\label{SecII}


In this section, we describe our compact objets model and we recall the partial wave expansions of the differential scattering cross section for plane monochromatic scalar waves impinging upon a compact body and we apply the CAM-approach developed in \cite{Folacci:2019cmc} and \cite{Folacci:2019vtt}

\subsection{Our model}
\label{SecIIa}

  We consider that the gravitating source a spherically symmetric incompressible perfect fluid ball of uniform
density (see Ref~\cite{Dolan:2017rtj} and reference therein) in a coordinate system $\{t,r,\theta,\varphi\}$ and is defined by the metric
\begin{equation}\label{Line_elem}
  ds^2= -f(r) dt^2+h(r)^{-1}dr^2+r^2d\sigma_2^2
\end{equation}
where $d\sigma_2^2$ denotes the metric on the unit 2-sphere $S^2$. Here, the radial function $h(r)$ is continuous but not differentiable across the surface of the star $r=R$ and the function $f(r)$ is one, but not twice, differentiable. We recall that, in the exterior of the star (i.e., $r>R$), we have $f(r)=h(r)=a-2M/r$ by Birkhoff's~\cite{VojeJohansen:2005nd}. In the interior of the star (i.e. $r<R$), we have the Schwarzschild solution for an incompressible fluid~\cite{Shapiro1983}

\begin{subequations}\label{Interior_Solution}
\begin{align}\label{Interior_Solution_f}
    f(r) =\frac{1}{4}\left(1-\frac{2 M r^2}{R^3}\right)+\frac{9}{4}\left(1-\frac{2M}{R}\right) \nonumber\\ -\frac{3}{2}
                \sqrt{\left(1-\frac{2M}{R}\right)\left(1-\frac{2 M r^2}{R^3}\right)}
\end{align}
\begin{equation}\label{Interior_Solution_h}
 h(r) =1-\frac{2 M r^2}{R^3}
\end{equation}
\end{subequations}


\subsection{Partial wave expansions of differential scattering cross sections}
\label{SecIIb}

We recall that, for the scalar field, the differential scattering cross section is given by (see, e.g.,\cite{Dolan:2017rtj} and references therein)
\begin{equation}\label{Scalar_Scattering_diff}
  \frac{d\sigma}{d\Omega} = |\hat{f}(\omega,\theta)|^2
\end{equation}
where
\begin{equation}\label{Scalar_Scattering_amp}
 \hat{f}(\omega,\theta) = \frac{1}{2 i \omega} \sum_{\ell = 0}^{\infty} (2\ell+1)[S_{\ell}(\omega)-1]P_{\ell}(\cos\theta)
\end{equation}
denotes the scattering amplitude.  In Eqs.~(\ref{Scalar_Scattering_amp}), the functions $P_{\ell}(\cos\theta)$ are the Legendre polynomials \cite{AS65}.  We also recall that the $S$-matrix elements $S_{\ell}(\omega)$ appearing in Eqs.~(\ref{Scalar_Scattering_amp}) can be defined from the modes $\phi_{\omega \ell}^{in}$ solutions of the homogenous radial equation
\begin{equation}
\label{H_Radial_equation}
\left[\frac{d^{2}}{dr_{\ast}^{2}}+\omega^{2}-V_{\ell}(r)\right]\phi_{\omega\ell}= 0
\end{equation}
here $dr/dr^\ast =\sqrt{f(r)h(r)}$ denotes the tortoise coordinate, where
\begin{equation}\label{Inside_Potentiel}
  V_{\ell}(r) =f(r)\left[\frac{\ell(\ell+1)}{r^2}+\frac{h(r)}{2r}\left(\frac{f'(r)}{f(r)}+\frac{h'(r)}{h(r)}\right)\right]
\end{equation}
the potential inside the star ($r<R$) and  $V_{\ell}(r)$ reduces to the Regge-Wheeler potential outside the star ($r>R$)
\begin{equation}\label{RW_Potentiel}
  V_{\ell}(r) =\left(1-\frac{2M}{r}\right)\left(\frac{\ell(\ell+1)}{r^2}+\frac{2M}{r^3}\right)
\end{equation}
and the tortoise coordinates $r^\ast$ reduces to $r^\ast = r+ 2M \ln [r/(2M) -1]+\mathrm{const}$.

It is important to recall that the modes $\phi_{\omega \ell}^{in}$ have a regular behavior at the origin $r \to 0$
\begin{equation}\label{bc_1_in}
\phi^\mathrm{in}_{\omega  \ell}(r) \scriptstyle{\underset{r \to 0}{\sim}}
\displaystyle{r^{\ell+1}}.
\end{equation}
of the compact body and an asymptotic behavior at spatial infinity $r \to +\infty$ (i.e., for $r_\ast \to +\infty$) of the form
\begin{equation}\label{bc_2_in}
\phi^\mathrm{in}_{\omega  \ell}(r) \scriptstyle{\underset{r_\ast \to +\infty}{\sim}}
\displaystyle{ A^{(-)}_\ell (\omega) e^{-i\omega r_\ast} + A^{(+)}_\ell (\omega) e^{+i\omega r_\ast}}.
\end{equation}
In this last equation, the coefficients $A^{(-)}_\ell (\omega)$ and  $A^{(+)}_\ell (\omega)$ are complex amplitudes and we have
\begin{equation}\label{Matrix_S}
  S_{\ell}(\omega) =  e^{i(\ell+1)\pi} \, \frac{A_{\ell}^{(+)}(\omega)}{A_{\ell}^{(-)}(\omega)}.
\end{equation}

\subsection{CAM representation of the scattering amplitude}
\label{SecIIc}

To construct the CAM representation, we follow the steps in section~II of the Ref~\cite{Folacci:2019cmc} and recall the main results.


By using the Sommerfeld-Watson transformation \cite{Watson18,Sommerfeld49,Newton:1982qc} which permits us to write
\begin{equation}\label{SWT_gen}
\sum_{\ell=0}^{+\infty} (-1)^\ell F(\ell)= \frac{i}{2} \int_{\cal C} d\lambda \, \frac{F(\lambda -1/2)}{\cos (\pi \lambda)}
\end{equation}
with a function $F$ without any singularities on the real $\lambda$ axis,we replace in Eq.~(\ref{Scalar_Scattering_amp}) the discrete sum over the ordinary angular momentum $\ell$ by a contour integral in the complex $\lambda$ plane (i.e., in the complex $\ell$ plane with $\lambda = \ell +1/2$). By noting that $P_\ell (\cos \theta)=(-1)^\ell P_\ell (-\cos \theta)$, we obtain
\begin{eqnarray}\label{SW_Scalar_Scattering_amp}
& & \hat{f}(\omega,\theta) = \frac{1}{2 \omega}  \int_{\cal C} d\lambda \, \frac{\lambda}{\cos (\pi \lambda)} \nonumber \\
&&  \qquad\qquad   \times \left[ S_{\lambda -1/2} (\omega) -1 \right]P_{\lambda -1/2} (-\cos \theta).
\end{eqnarray}
In Eqs.~(\ref{SWT_gen}) and (\ref{SW_Scalar_Scattering_amp}), the integration contour encircles counterclockwise the positive real axis of the complex $\lambda$ plane, i.e., we take ${\cal C}=]+\infty +i\epsilon,+i\epsilon] \cup
[+i\epsilon,-i\epsilon] \cup [-i\epsilon, +\infty -i\epsilon[$ with $\epsilon \to 0_+$ (see Fig.1 Ref~\cite{Folacci:2019cmc}).

Here, we note that, the Legendre function of first kind $P_{\lambda -1/2} (z)$ denotes the analytic extension of the Legendre polynomials $P_\ell (z)$. It is defined in terms of hypergeometric functions by \cite{AS65}
\begin{equation}\label{Def_ext_LegendreP}
P_{\lambda -1/2} (z) = F[1/2-\lambda,1/2+\lambda;1;(1-z)/2].
\end{equation}
In Eq.~(\ref{SW_Scalar_Scattering_amp}), $S_{\lambda -1/2} (\omega)$ denotes ``the'' analytic extension of $S_\ell (\omega)$. It is given by [see Eq.~(\ref{Matrix_S})]
\begin{equation}\label{Matrix_S_CAM}
  S_{\lambda -1/2}(\omega) =  e^{i(\lambda + 1/2)\pi} \, \frac{A_{\lambda -1/2}^{(+)}(\omega)}{A_{\lambda -1/2}^{(-)}(\omega)}
\end{equation}
where the complex amplitudes $A^{(-)}_{\lambda -1/2} (\omega)$ and  $A^{(+)}_{\lambda -1/2} (\omega)$ are defined from the analytic extension of the modes $\phi_{\omega \ell}^{in}$, i.e., from the function $\phi_{\omega ,\lambda -1/2}^{in}$.

It is important to note that the poles of  $S_{\lambda -1/2} (\omega)$ in the complex $\lambda$ plan (i.e., the Regge poles) are defined  as  the zeros $\lambda_n(\omega)$ with $n=1,2,3,\ldots$ of the coefficient  $A^{(-)}_{\lambda-1/2} (\omega)$ [see Eq.~(\ref{Matrix_S_CAM})]
\begin{equation}\label{PR_def_Am}
A^{(-)}_{\lambda_n(\omega)-1/2} (\omega)=0.
\end{equation}
and the associated residues will play a central role. We note that the residue of the matrix $S_{\lambda-1/2}(\omega)$ at the pole $\lambda=\lambda_n(\omega)$ is defined by [see Eq.~(\ref{Matrix_S_CAM})]
\begin{equation}\label{residues_RP}
r_n(\omega)=e^{i\pi [\lambda_n(\omega)+1/2]} \left[ \frac{A_{\lambda -1/2}^{(+)}(\omega)}{\frac{d}{d \lambda}A_{\lambda -1/2}^{(-)}(\omega)}\right]_{\lambda=\lambda_n(\omega)}.
\end{equation}


Now, we ``deform'' the contour ${\cal C}$ in Eq.~(\ref{SW_Scalar_Scattering_amp}) in order to collect, by using the Cauchy's theorem, the Regge poles contributions. This achieved by following, \textit{mutatis mutandis}, the approach developed in Ref~\cite{Folacci:2019cmc} (see more particularly Sec. IIB~3 and Fig.~1). We obtain

\begin{equation}\label{CAM_Scalar_Scattering_amp_tot}
\hat{f} (\omega, \theta) =  \hat{f}^\text{\tiny{B}} (\omega, \theta) +  \hat{f}^\text{\tiny{RP}} (\omega, \theta)
\end{equation}
where
\begin{subequations}\label{CAM_Scalar_Scattering_amp_decomp}
\begin{equation}\label{CAM_Scalar_Scattering_amp_decomp_Background}
\hat{f}^\text{\tiny{B}} (\omega, \theta) = \hat{f}^\text{\tiny{B},\tiny{Re}} (\omega, \theta)+ \hat{f}^\text{\tiny{B},\tiny{Im}} (\omega, \theta)
\end{equation}
with
\begin{equation}\label{CAM_Scalar_Scattering_amp_decomp_Background_a}
\hat{f}^\text{\tiny{B},\tiny{Re}} (\omega, \theta) = \frac{1}{\pi \omega} \int_{{\cal C}_{-}} d\lambda \, \lambda S_{\lambda -1/2}(\omega) Q_{\lambda -1/2}(\cos \theta +i0)
\end{equation}
and
\begin{eqnarray}\label{CAM_Scalar_Scattering_amp_decomp_Background_b}
\hat{f}^\text{\tiny{B},\tiny{Im}} && (\omega, \theta) = \frac{1}{2 \omega}\left(\int_{+i\infty}^{0} d\lambda \, \left[S_{\lambda -1/2}(\omega) P_{\lambda_n(\omega) -1/2} (-\cos \theta) \right. \right.\nonumber \\
&& -\left. \left. S_{-\lambda -1/2}(\omega) e^{i \pi \left(\lambda+1/2\right)}P_{\lambda_n(\omega) -1/2} (\cos \theta) \right]\lambda\right)
\end{eqnarray}
\end{subequations}
is a background integral contribution and where
\begin{eqnarray}\label{CAM_Scalar_Scattering_amp_decomp_RP}
& & \hat{f}^\text{\tiny{RP}} (\omega, \theta) = -\frac{i \pi}{\omega}    \sum_{n=1}^{+\infty}   \frac{ \lambda_n(\omega) r_n(\omega)}{\cos[\pi \lambda_n(\omega)]}  \nonumber \\
&&  \qquad\qquad \qquad\qquad \times  P_{\lambda_n(\omega) -1/2} (-\cos \theta)
\end{eqnarray}
is a sum over the Regge poles lying in the first quadrant of the CAM plane. Of course, Eqs.~(\ref{CAM_Scalar_Scattering_amp_tot}), (\ref{CAM_Scalar_Scattering_amp_decomp}) and (\ref{CAM_Scalar_Scattering_amp_decomp_RP}) provide an exact representation of the scattering amplitude $\hat{f} (\omega, \theta)$ for the scalar field, equivalent to the initial partial wave expansion (\ref{Scalar_Scattering_amp}). From this CAM representation, we can extract the contribution $\hat{f}^\text{\tiny{RP}} (\omega, \theta)$ given by (\ref{CAM_Scalar_Scattering_amp_decomp_RP}) which, as a sum over Regge poles, is only an approximation of $\hat{f} (\omega, \theta)$, and which provides us with an approximation of the differential scattering cross section (\ref{Scalar_Scattering_diff}).

\section{Reconstruction of differential scattering cross sections from Regge pole sums}
\label{SecIII}

In this section,  we compare CAM representations or, more precisely, their Regge pole approximations with their partial wave expansions of the differential scattering cross sections.

\subsection{Computational methods}
\label{SecIIIa}

To construct the scattering amplitude \eqref{Scalar_Scattering_amp}, the back ground integrals~\eqref{CAM_Scalar_Scattering_amp_decomp_Background_a} and~\eqref{CAM_Scalar_Scattering_amp_decomp_Background_b} as well as the Regge pole contribution~\eqref{CAM_Scalar_Scattering_amp_decomp_RP}, we use, \textit{mutatis mutandis} the computational methods that have permitted us, in the Refs~\cite{Folacci:2019cmc,Folacci:2019vtt}, to reconsider the CAM representation for scattering of the scalar, electromagnetic and gravitational waves by Schwarzschild BH (see also Ref~\cite{Dolan:2017rtj}). It is important to remark that, duo to the long rang nature of the field propagating on the Schwarzschild BH (outside the compact body), the scattering amplitude~\eqref{Scalar_Scattering_amp} and the background integral~\eqref{CAM_Scalar_Scattering_amp_decomp_Background_a} suffer a lack of convergence and to overcome this problem, i.e., to accelerate the convergence of these sum and integral, we have used the method described int the Appendix of Ref~\cite{Folacci:2019cmc}. It should be noted that we have performed all the numerical calculations by using {\it Mathematica} \cite{Mathematica}.


\subsection{Results and comments}
\label{SecIIIb}




 \begin{figure}[htb]
\centering
 \includegraphics[scale=0.50]{RP_R_6_2Mw_3_16}
\caption{\label{RP_approx_2Mw_3_6_s_1} The Regge poles $\lambda_n(\omega)$ for the scalar field. We assume $2M =1$}
\end{figure}


 \begin{figure}[htb]
\centering
 \includegraphics[scale=0.50]{RP_R_2_dot_26_2Mw_3_6}
\caption{\label{RP_approx_2Mw_3_6_s_1} The Regge poles $\lambda_n(\omega)$ for the scalar field. We assume $2M =1$}
\end{figure}


\begin{figure*}%[h!]
 \includegraphics[scale=0.50]{Scattering_Cross_Section_R_6_2Mw_3}
\caption{\label{S_0_2Mw_01_Exact_vs_CAM} The scalar cross section of a compact bodies for $2M\omega=3$ and $R=6M$, its Regge pole approximation and the background integral contribution.}
\end{figure*}

\begin{figure*}%[h!]
 \includegraphics[scale=0.50]{Scattering_Cross_Section_R_6_2Mw_16}
\caption{\label{S_0_2Mw_01_Exact_vs_CAM} The scalar cross section of a compact bodies for $2M\omega=16$ and $R=6M$, its Regge pole approximation and the background integral contribution.}
\end{figure*}

\begin{figure*}%[h!]
\centering
 \includegraphics[scale=0.50]{Scattering_Cross_Section_R_2-dot-26_2Mw_3}
\caption{\label{S_0_2Mw_06_Exact_vs_CAM} The scalar cross section of a very compact bodies for $2M\omega=3$ and $R=2.26M$ and its Regge pole approximation.}
\end{figure*}


\begin{figure*}%[h!]
\centering
 \includegraphics[scale=0.50]{Scattering_Cross_Section_R_2-dot-26_2Mw_6}
\caption{\label{S_0_2Mw_06_Exact_vs_CAM} The scalar cross section of a very compact bodies for $2M\omega=6$ and $R=2.26M$ and its Regge pole approximation.}
\end{figure*}


\begin{figure}%[h!]
\centering
 \includegraphics[scale=0.50]{Rainbow_Cross_Section_R_6_2Mw_16}
\caption{\label{S_0_2Mw_06_Exact_vs_CAM} Rainbow scattering for compact bodies for $2M\omega=16$ and $R=6M$, its Regge pole approximation and different contributions of the sum over Regge poles.}
\end{figure}


\begin{figure}%[h!]
\centering
 \includegraphics[scale=0.50]{Diff_Contribution_Cross_Section_R_2-dot-26_2Mw_6}
\caption{\label{S_0_2Mw_06_Exact_vs_CAM} Rainbow scattering for compact bodies for $2M\omega=16$ and $R=2.26M$, its Regge pole approximation and different contributions of the sum over Regge poles.}
\end{figure}


\begingroup
\squeezetable
\begin{table*}[htp]
\begin{threeparttable}[htp]
%\captionsetup{font=small}
\caption{\label{tab:table2} The lowest Regge poles $\lambda_{n}(\omega)$ for the scalar field and the associated residues $r_{n}(\omega)$. The radius of the compact bodies is $R = 6M$ and we assume $2M=1$.}
\smallskip
\centering
\begin{ruledtabular}
\begin{tabular}{cccccc}
 $n$ & $\omega$  & $\lambda^{\text{(S-W)\tnote{1}}}_n(\omega)$ & $\lambda^{\text{(B-R)\tnote{2}}}_n(\omega)$ & $r^{\text{(S-W)}}_{n}(\omega)$ & $r^{\text{(B-R)}}_{n}(\omega)$
 \\ \hline
$1$  & $3$  & $ 9.64850+2.76784 i$  & $1.56219+2.33072 i$  & $-12.41483-0.10424 i$  & $ -0.184457+0.480330 i$    \\
     & $16$  & $56.00945+5.71038 i$  & $ 0.62529+3.27098 i$  & $-447.5395+25.2912 i$  & $-0.322061-0.088002 i $  \\

$2$  & $3$  & $ 10.71986+5.16209 i $  & $3.81484+2.48159 i$  & $13.8486+24.3824 i$  & $0.290952+1.043116 i$    \\
     & $16$  & $ 58.442656+9.18793 i$  & $3.14868+3.31439 i$  & $5188.750-859.909 i$  & $ -0.381581-0.077583 i$    \\

$3$  & $3$  & $11.62296+7.17454 i$  & $ 6.35675+2.64104 i$  & $39.4189-12.3554 i$ & $2.83038-0.28686 i$    \\
     & $16$  & $60.20374+12.14965 i$  & $4.70011+3.35821 i$  & $-29331.71-18578.38 i$  & $-0.456423-0.021249 i$ \\

$4$  & $3$  & $ 12.4297+8.9960 i$  & $/$  & $ 13.2301-50.8802 i$ & $/$    \\
     & $16$  & $ 61.67700+14.84728 i $  & $6.78093+3.40257 i$  & $-15868.9+161199.9 i$  & $-0.528929+0.106794 i$ \\

$5$  & $3$  & $13.1734+10.6929 i$  & $/$  & $-33.7366-51.7404 i$ & $/$     \\
     & $16$  & $62.98626+17.37165 i$  & $8.89270+3.44762 i$  & $589920.5-79507.8 i$  & $-0.550038+0.330275 i$    \\

$6$  & $3$  & $13.8709+12.2989 i$  & $/$   & $-66.4436-20.7767 i$ & $/$    \\
     & $16$  & $ 64.18605+19.76911 i$  & $11.03720+3.49356 i$  & $-360464.-1.797518\times 10^6 i$  & $ -0.426365+0.639191 i$    \\

$7$  & $3$  & $14.5322+13.8342 i$  & $/$   & $-73.0825+21.9088 i$ & $/$     \\
     & $16$  & $65.30640+22.06743 i$  & $13.21653+3.54058 i$  & $-4.880638\times 10^6+646112. i$  & $-0.038292+0.926498 i$    \\

$8$  & $3$  & $15.1640+15.3122 i$  & $/$   & $-56.3641+59.6187 i$ & $/$     \\
     & $16$  & $66.36581+24.28491 i$  & $15.4331+3.5889 i$  & $-479098.+1.1836070\times 10^7 i$  & $0.652285+0.920876 i$    \\

$9$  & $3$  & $15.7709+16.7425 i$  & $/$   & $-25.0183+83.3731 i$ & $/$     \\
     & $16$  & $67.37659+26.43447 i$  & $17.6898+3.6390 i$  & $2.487209\times 10^7+7.72797\times 10^6 i$  & $1.363464+0.248276 i$    \\

$10$  & $3$  & $16.3565+18.1321 i$  & $/$   & $11.7631+90.6815 i$ & $/$     \\
     & $16$  & $68.34738+28.52564 i $  & $19.9900+3.6910 i$  & $3.163822\times 10^7-4.265475\times 10^7 i$  & $1.29469-1.13096 i$    \\
\end{tabular}
\end{ruledtabular}
\begin{tablenotes}
     \item[1] Surface waves
     \item[2] Broad resonances
   \end{tablenotes}
\end{threeparttable}
\end{table*}
\endgroup


\begingroup
\squeezetable
\begin{table*}[htp]
\begin{threeparttable}[htp]
%\captionsetup{font=small}
\caption{\label{tab:table2} The lowest Regge poles $\lambda_{n}(\omega)$ for the scalar field and the associated residues $r_{n}(\omega)$. The radius of the compact bodies is $R = 2.26M$ and we assume $2M=1$.}
\smallskip
\centering
\begin{ruledtabular}
\begin{tabular}{cccccccc}
 $n$ & $\omega$  & $\lambda^{\text{(S-W)\tnote{1}}}_n(\omega)$  & $\lambda^{\text{(B-R)\tnote{2}}}_n(\omega)$ & $\lambda^{\text{(N-R)\tnote{3}}}_n(\omega)$ & $r^{\text{(S-W)}}_{n}(\omega)$ & $r^{\text{(B-R)}}_{n}(\omega)$ & $r^{\text{(N-R)}}_{n}(\omega)$
 \\ \hline
$1$  & $3$  & $5.871590+1.553799 i$  & $1.73455+1.64951 i  $  & $ 6.48474+0.68765 i $  & $-179.7945+131.4187 i $ & $ -1.52081-2.30968 i$ & $-2.5672-15.3797 i $  \\
     & $6$  & $12.991923+1.754967 i $  & $ 1.89664+2.13696 i  $  & $ 13.34118+1.13496 i $  & $4356.193+647.790 i $ & $  -0.66176-1.31963 i$ & $  -390.218+379.906 i $  \\

$2$  & $3$  & $5.778805+3.228990 i  $  & $ 3.48084+1.45765 i$  & $  7.25606+0.24457 i$  & $428.6893-235.0321 i $ & $16.2123+5.2371 i $ & $ -0.272250-1.150335 i$  \\
     & $6$  & $12.705495+3.383881 i $  & $ 3.74238+2.01309 i $  & $14.18757+0.68182 i $  & $-35075.99-9772.94 i $ & $-2.93679+4.83548 i $ & $ -11.3519+34.5571 i $  \\

$3$  & $3$  & $ 5.924546+4.705899 i $  & $  5.10229+1.29099 i$  & $ 7.95763+0.01764 i$  & $-404.6185-390.8531 i$ & $ 70.4849+54.1888 i  $ & $ -0.0370202-0.0048174 i $  \\
     & $6$  & $12.596259+4.982661 i $  & $  5.49829+1.89576 i $  & $14.9017+0.2912 i $  & $82360.19+81990.53 i$ & $ 6.7872-16.9564 i$ & $0.27028+2.27905 i  $  \\

$4$  & $3$  & $ 6.144986+6.043188 i $  & $ /$  & $/ $  & $ -471.5443+314.3116 i  $ & $ /$ & $/ $  \\
     & $6$  & $ 12.614598+6.503749 i  $  & $ 7.17509+1.78279 i $  & $15.5621+0.0422 i  $  & $39281.5-229393.2 i  $ & $39.6176+33.5152 i  $ & $0.1011154+0.0020569 i $  \\

$5$  & $3$  & $6.398427+7.281723 i $  & $ /$  & $/ $  & $37.8777+546.8945 i $ & $ /$ & $/ $  \\
     & $6$  & $  12.71646+7.95208 i $  & $ 8.78112+1.67243 i  $  & $ /$  & $ -356055.5+34945.9 i$ & $ 2.1175+134.5962 i $ & $/ $  \\

$6$  & $3$  & $ 6.666837+8.447532 i $  & $/ $  & $ /$  & $418.7890+315.4209 i$ & $/ $ & $/ $  \\
     & $6$  & $ 12.87420+9.33552 i $  & $ 10.32300+1.56317 i$  & $/ $  & $45934.6+468157.5 i $ & $ 66.944+324.598 i$ & $/ $  \\

$7$  & $3$  & $ 6.941642+9.557619 i $  & $ /$  & $/ $  & $499.2703-37.6476 i $ & $ /$ & $ /$  \\
     & $6$  & $13.06993+10.66226 i $  & $ 11.80630+1.45720 i  $  & $ /$  & $558619.4+61956.5 i $ & $ 833.855+78.332 i $ & $/ $  \\

$8$  & $3$  & $7.218463+10.623548 i $  & $ /$  & $/ $  & $ 367.2578-307.7533 i  $ & $ /$ & $/ $  \\
     & $6$  & $13.29184+11.93979 i $  & $/ $  & $/ $  & $ 293571.8-559756.8 i $ & $/ $ & $/ $  \\

$9$  & $3$  & $ 7.494953+11.653498 i $  & $/ $  & $/ $  & $147.3038-435.8160 i  $ & $ /$ & $/ $  \\
     & $6$  & $ 13.53197+13.17461 i  $  & $/ $  & $/ $  & $ -376511.5-570254.0 i  $ & $/$ & $ /$  \\

$10$ & $3$  & $7.76982+12.65345 i $  & $/ $  & $/ $  & $-71.8294-437.3469 i  $ & $/ $ & $ /$  \\
     & $6$  & $ 13.78485+14.37216 i$  & $ /$  & $/ $  & $ -719306.1-20011.7 i $ & $ /$ & $/ $  \\

\end{tabular}
\end{ruledtabular}
\begin{tablenotes}
     \item[1] Surface waves
     \item[2] Broad resonances
     \item[3] Narrow  resonances
   \end{tablenotes}
\end{threeparttable}
\end{table*}
\endgroup


\section{Classification of quasinormal modes}

The study of stellar quasinormal modes began with a general relativistic extension of the Newtonian treatment of fluid pulsations of stars. Newly formed neutron stars, the remnants of supernovae collapse, are predicted to pulsate with a large initial energy \cite{thorne1967non}. It was and still is of interest to study how these pulsations lose energy in the form of gravitational waves. Recently, the gravitational waves from a neutron star merger were detected... In particular the emitted radiation should cary a signature of the modes of oscillation allowed inside the star. These fluid modes can be classified in analogy to their real Newtonian counterparts, with an additional damping time due to emission of GWs. Later, Kokkotas and Schutz showed that an additional family of modes existed \cite{Kokkotas:1986gd,Kokkotas:1992ka}, dubbed $\omega$-modes. These modes are characterised by a negligible excitation of fluid motion (and in the axial sector no fluid motion). They're highly damped and correspond to excitations of the dynamical perturbed space-time. For an excellent review of (gravitational) quasinormal modes in relativistic stars and black holes see \cite{Kokkotas:1999bd}.

We consider the massless scalar field which only couples to the body via gravitation. The scalar modes obey an equation of motion very similar to the axial gravitational wave sector. Accordingly, a similar spectrum for scalar modes can be expected (with no fluid excitation). Whilst these are arguably of less astrophysical interest, they are nonetheless a good model for axial GWs and allow us to make a first step towards a CAM approach for perturbed relativistic stars. We will classify the scalar $\omega$-modes as the axial GW modes have been, we have

	\begin{enumerate}
		\item Curvature modes: These modes exist for all relativistic stars. They are rapidly damped, and the damping is larger for less compact bodies.
		\item Trapped modes: These are the scalar analogue of the axial trapped modes considered by Chandrasekhar and Ferrari \cite{Chandrasekhar449}. They exist for bodies with $R/M < 3$. These models have an effective radial potential with a local minimum somewhere inside the star, and local maximum at the photon sphere $r=3M$. The trapped modes are essentially the first few curvature modes with energy less than the potential barrier. There are a finite number of them, and the number increases with the depth of the potential well.
		\item Interface modes: also known as $\omega_{\text{II}}$-modes, these were discovered by Leins \textit{et al.} \cite{Leins:1993zz}. They used a generalisation of Leaver's continued fraction  method as well as a `Wronskain method' that allowed them to accurately compute these modes which are characterised by very rapid damping (large imaginary part). This second branch of modes is the most similar to the Schwarzschild black hole quasinormal modes. We have taken their continued fraction method one step further (see section...) in order to be confident with our results.
	\end{enumerate}


\acknowledgments
M. O. E. H. wish to thank Antoine Folacci for various discussions concerning this work. T.S.~acknowledges financial support from EPSRC. S.R.D.~acknowledges financial support from the European Union's Horizon 2020 research and innovation programme under the H2020-MSCA-RISE-2017 Grant No.~FunFiCO-777740, and from the Science and Technology Facilities Council (STFC) under Grant No.~ST/P000800/1.



\bibliography{rainbowCAM}
\bibliographystyle{apsrev4-1}

\end{document}
