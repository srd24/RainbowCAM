 \documentclass[preprint,showpacs,preprintnumbers,amssymb,superscriptaddress,aps,prd,nofootinbib,11pt]{revtex4-1}
\linespread{1.2}


\usepackage{graphicx, epsfig, amssymb} 
\usepackage{amsmath, amsfonts}
\usepackage{bm} 
\usepackage{enumitem}
\usepackage[usenames]{color}
\definecolor{navyblue}{rgb}{0.0, 0.0, 0.5}
\usepackage[linktocpage,colorlinks=true,allcolors=navyblue]{hyperref}
\usepackage[caption=false]{subfig}
\usepackage[utf8]{inputenc}
\usepackage{tensor}
\usepackage{soul}
%\usepackage{enumerate}
\usepackage{pict2e}
\usepackage{diagbox}
\usepackage{empheq}

%\usepackage[margin=1in]{geometry}
\usepackage{float}
%\usepackage{subfig}
\usepackage{comment}
\usepackage{array}
%\usepackage{varioref}
%\usepackage[hidelinks]{hyperref}
%\usepackage{cleveref}
%\usepackage[numbers]{natbib}
\usepackage{units}
\usepackage{wasysym}
%\usepackage{caption}
%\usepackage{subcaption}

\DeclareMathAlphabet{\mathpzc}{OT1}{pzc}{m}{it}
\newcommand{\ff}{\hat{f}}

\def\l{\left}
\def\r{\right}
\def\p{\partial}

\newcommand{\angarg}{(\theta,\phi)}
\newcommand{\dth}{\frac{d}{d \theta}}
\newcommand{\sh}{ {}_{-s} S_{\ell m}^{a \omega}}
\newcommand{\zh}{ {}_{-s} Z_{\ell m}^{a \omega}}
\newcommand{\syjm}{  {}_{-s}Y_{jm} }
\newcommand{\sylm}{  {}_{-s} Y_{\ell m} }
\newcommand{\sumjm}{\sum_{j=s}^{\infty} \sum_{m=-j}^{j}}
\newcommand{\sumjms}{\sum_{j,m}}
\newcommand{\cS}{\mathcal{S}}
\newcommand{\lmws}{{\ell m \omega s}}
\newcommand{\nn}{\nonumber}

\newcommand*\widefbox[1]{\fbox{\rule[-2cm]{0pt}{4cm}\hspace{2em}#1\hspace{2em}}}

\newcommand{\sam}[1]{\textcolor{red}{(Sam: #1)}}
\newcommand{\tom}[1]{\textcolor{blue}{(Tom: #1)}}
\newcommand{\mo}[1]{\textcolor{green}{(Mohamed: #1)}}

\begin{document}

\title{Scattering from compact objects, Regge poles \\ and the Complex Angular Momentum method}
 
\author{Mohamed Ould Elhadj}\email{m.ouldelhadj@sheffield.ac.uk}
\affiliation{Consortium for Fundamental Physics,  School of Mathematics and Statistics,
University of Sheffield, Hicks Building, Hounsfield Road, Sheffield S3 7RH, United Kingdom \looseness=-1}
%
\author{Tom Stratton}\email{tstratton1@sheffield.ac.uk}
\affiliation{Consortium for Fundamental Physics,  School of Mathematics and Statistics,
University of Sheffield, Hicks Building, Hounsfield Road, Sheffield S3 7RH, United Kingdom \looseness=-1}
%
\author{Sam R. Dolan}\email{s.dolan@sheffield.ac.uk}
\affiliation{Consortium for Fundamental Physics,  School of Mathematics and Statistics,
University of Sheffield, Hicks Building, Hounsfield Road, Sheffield S3 7RH, United Kingdom \looseness=-1}
% 
\begin{abstract}
To be written
\end{abstract}

\date{\today}

\maketitle


\section{Introduction}
Blah. hello. some more text.

 The topic of time-independent scattering has been studied in detail since the 1960s \cite{Hildreth1964PhDT64, Matzner:1968, Vishveshwara:1970}, and there now exists a substantial literature \cite{Mashhoon:1973zz,Chrzanowski:1976jb,DeLogi:1977dp,Sanchez:1977vz,MatznerRyan1978,Handler:1980un,Matzner:1985rjn,Futterman:1988ni,Andersson:1995vi,Glampedakis:2001cx,Dolan:2006vj,Dolan:2007ut,Dolan:2008kf,Crispino:2009xt,Cotaescu:2014jca,Sorge:2015yoa,Gussmann:2016mkp,Leite:2017zyb,Nambu:2019sqn, Folacci:2019vtt,Leite:2019eis}. 




\section{Classification of quasinormal modes}

The study of stellar quasinormal modes began with a general relativistic extension of the Newtonian treatment of fluid pulsations of stars. Newly formed neutron stars, the remnants of supernovae collapse, are predicted to pulsate with a large initial energy \cite{thorne1967non}. It was and still is of interest to study how these pulsations lose energy in the form of gravitational waves. Recently, the gravitational waves from a neutron star merger were detected... In particular the emitted radiation should cary a signature of the modes of oscillation allowed inside the star. These fluid modes can be classified in analogy to their real Newtonian counterparts, with an additional damping time due to emission of GWs. Later, Kokkotas and Schutz showed that an additional family of modes existed \cite{Kokkotas:1986gd,Kokkotas:1992ka}, dubbed $\omega$-modes. These modes are characterised by a negligible excitation of fluid motion (and in the axial sector no fluid motion). They're highly damped and correspond to excitations of the dynamical perturbed space-time. For an excellent review of (gravitational) quasinormal modes in relativistic stars and black holes see \cite{Kokkotas:1999bd}. 

We consider the massless scalar field which only couples to the body via gravitation. The scalar modes obey an equation of motion very similar to the axial gravitational wave sector. Accordingly, a similar spectrum for scalar modes can be expected (with no fluid excitation). Whilst these are arguably of less astrophysical interest, they are nonetheless a good model for axial GWs and allow us to make a first step towards a CAM approach for perturbed relativistic stars. We will classify the scalar $\omega$-modes as the axial GW modes have been, we have 

	\begin{enumerate}
		\item Curvature modes: These modes exist for all relativistic stars. They are rapidly damped, and the damping is larger for less compact bodies. 
		\item Trapped modes: These are the scalar analogue of the axial trapped modes considered by Chandrasekhar and Ferrari \cite{Chandrasekhar449}. They exist for bodies with $R/M < 3$. These models have an effective radial potential with a local minimum somewhere inside the star, and local maximum at the photon sphere $r=3M$. The trapped modes are essentially the first few curvature modes with energy less than the potential barrier. There are a finite number of them, and the number increases with the depth of the potential well. 
		\item Interface modes: also known as $\omega_{\text{II}}$-modes, these were discovered by Leins \textit{et al.} \cite{Leins:1993zz}. They used a generalisation of Leaver's continued fraction  method as well as a `Wronskain method' that allowed them to accurately compute these modes which are characterised by very rapid damping (large imaginary part). This second branch of modes is the most similar to the Schwarzschild black hole quasinormal modes. We have taken their continued fraction method one step further (see section...) in order to be confident with our results. 
	\end{enumerate}


\acknowledgments
T.S.~acknowledges financial support from EPSRC. S.R.D.~acknowledges financial support from the European Union's Horizon 2020 research and innovation programme under the H2020-MSCA-RISE-2017 Grant No.~FunFiCO-777740, and from the Science and Technology Facilities Council (STFC) under Grant No.~ST/P000800/1.







\bibliography{rainbowCAM}
\bibliographystyle{apsrev4-1}

\end{document}
