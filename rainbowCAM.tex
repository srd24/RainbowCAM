 \documentclass[preprint,showpacs,preprintnumbers,amssymb,superscriptaddress,aps,prd,nofootinbib,11pt]{revtex4-1}
\linespread{1.2}


\usepackage{graphicx, epsfig, amssymb} 
\usepackage{amsmath, amsfonts}
\usepackage{bm} 
\usepackage{enumitem}
\usepackage[usenames]{color}
\definecolor{navyblue}{rgb}{0.0, 0.0, 0.5}
\usepackage[linktocpage,colorlinks=true,allcolors=navyblue]{hyperref}
\usepackage[caption=false]{subfig}
\usepackage[utf8]{inputenc}
\usepackage{tensor}
\usepackage{soul}
%\usepackage{enumerate}
\usepackage{pict2e}
\usepackage{diagbox}
\usepackage{empheq}

%\usepackage[margin=1in]{geometry}
\usepackage{float}
%\usepackage{subfig}
\usepackage{comment}
\usepackage{array}
%\usepackage{varioref}
%\usepackage[hidelinks]{hyperref}
%\usepackage{cleveref}
%\usepackage[numbers]{natbib}
\usepackage{units}
\usepackage{wasysym}
%\usepackage{caption}
%\usepackage{subcaption}

\DeclareMathAlphabet{\mathpzc}{OT1}{pzc}{m}{it}
\newcommand{\ff}{\hat{f}}

\def\l{\left}
\def\r{\right}
\def\p{\partial}

\newcommand{\angarg}{(\theta,\phi)}
\newcommand{\dth}{\frac{d}{d \theta}}
\newcommand{\sh}{ {}_{-s} S_{\ell m}^{a \omega}}
\newcommand{\zh}{ {}_{-s} Z_{\ell m}^{a \omega}}
\newcommand{\syjm}{  {}_{-s}Y_{jm} }
\newcommand{\sylm}{  {}_{-s} Y_{\ell m} }
\newcommand{\sumjm}{\sum_{j=s}^{\infty} \sum_{m=-j}^{j}}
\newcommand{\sumjms}{\sum_{j,m}}
\newcommand{\cS}{\mathcal{S}}
\newcommand{\lmws}{{\ell m \omega s}}
\newcommand{\nn}{\nonumber}
\newcommand{\be}{\begin{equation}}
\newcommand{\ee}{\end{equation}}

\newcommand*\widefbox[1]{\fbox{\rule[-2cm]{0pt}{4cm}\hspace{2em}#1\hspace{2em}}}

\newcommand{\sam}[1]{\textcolor{red}{(Sam: #1)}}
\newcommand{\tom}[1]{\textcolor{blue}{(Tom: #1)}}
\newcommand{\mo}[1]{\textcolor{green}{(Mohamed: #1)}}

\begin{document}

\title{Scattering from compact objects, Regge poles \\ and the Complex Angular Momentum method}
 
\author{Mohamed Ould Elhadj}\email{m.ouldelhadj@sheffield.ac.uk}
\affiliation{Consortium for Fundamental Physics,  School of Mathematics and Statistics,
University of Sheffield, Hicks Building, Hounsfield Road, Sheffield S3 7RH, United Kingdom \looseness=-1}
%
\author{Tom Stratton}\email{tstratton1@sheffield.ac.uk}
\affiliation{Consortium for Fundamental Physics,  School of Mathematics and Statistics,
University of Sheffield, Hicks Building, Hounsfield Road, Sheffield S3 7RH, United Kingdom \looseness=-1}
%
\author{Sam R. Dolan}\email{s.dolan@sheffield.ac.uk}
\affiliation{Consortium for Fundamental Physics,  School of Mathematics and Statistics,
University of Sheffield, Hicks Building, Hounsfield Road, Sheffield S3 7RH, United Kingdom \looseness=-1}
% 
\begin{abstract}
To be written
\end{abstract}

\date{\today}

\maketitle


\section{Introduction}
Blah. hello. some more text.

 The topic of time-independent scattering has been studied in detail since the 1960s \cite{Hildreth1964PhDT64, Matzner:1968, Vishveshwara:1970}, and there now exists a substantial literature \cite{Mashhoon:1973zz,Chrzanowski:1976jb,DeLogi:1977dp,Sanchez:1977vz,MatznerRyan1978,Handler:1980un,Matzner:1985rjn,Futterman:1988ni,Andersson:1995vi,Glampedakis:2001cx,Dolan:2006vj,Dolan:2007ut,Dolan:2008kf,Crispino:2009xt,Cotaescu:2014jca,Sorge:2015yoa,Gussmann:2016mkp,Leite:2017zyb,Nambu:2019sqn, Folacci:2019vtt,Leite:2019eis}. 


\be
\cos \left( \tfrac{1}{2} \theta_R \right) = \frac{1}{n^2} \left[ \frac{4 - n^2}{3} \right]^{3/2}
\ee
(Huygens 1652). 


\acknowledgments
T.S.~acknowledges financial support from EPSRC. S.R.D.~acknowledges financial support from the European Union's Horizon 2020 research and innovation programme under the H2020-MSCA-RISE-2017 Grant No.~FunFiCO-777740, and from the Science and Technology Facilities Council (STFC) under Grant No.~ST/P000800/1.


\bibliography{rainbowCAM}
\bibliographystyle{apsrev4-1}

\end{document}
